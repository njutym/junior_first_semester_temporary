\documentclass[tikz,a4paper,UTF8]{article}
\usepackage{ctex}
\usepackage[margin=1.25in]{geometry}
\usepackage{color}
\usepackage{graphicx}
\usepackage{amssymb}
\usepackage{amsmath}
\usepackage{amsthm}
\usepackage{enumerate}
\usepackage{bm}
\usepackage{hyperref}
\usepackage{pgfplots}
\usepackage{epsfig}
\usepackage{color}
\usepackage{tcolorbox}
\usepackage{mdframed}
\usepackage{lipsum}
\usepackage{framed}
\usepackage{setspace}
\usepackage{hyperref}
\usepackage{pgfplots}
\pgfplotsset{compat=newest}

\newmdtheoremenv{thm-box}{myThm}
\newmdtheoremenv{prop-box}{Proposition}
\newmdtheoremenv{def-box}{定义}

\setlength{\evensidemargin}{.25in}
\setlength{\textwidth}{6in}
\setlength{\topmargin}{-0.5in}
\setlength{\topmargin}{-0.5in}
% \setlength{\textheight}{9.5in}
%%%%%%%%%%%%%%%%%%此处用于设置页眉页脚%%%%%%%%%%%%%%%%%%
\usepackage{fancyhdr}                                
\usepackage{lastpage}                                           
\usepackage{layout}                                             
\footskip = 10pt 
\pagestyle{fancy}                    % 设置页眉                 
\lhead{2023年秋季}                    
\chead{数字信号处理}                                                
% \rhead{第\thepage/\pageref{LastPage}页} 
\rhead{作业四}                                                                                               
\cfoot{\thepage}                                                
\renewcommand{\headrulewidth}{1pt}  			%页眉线宽,设为0可以去页眉线
\setlength{\skip\footins}{0.5cm}    			%脚注与正文的距离           
\renewcommand{\footrulewidth}{0pt}  			%页脚线宽,设为0可以去页脚线

\makeatletter 									%设置双线页眉                                        
\def\headrule{{\if@fancyplain\let\headrulewidth\plainheadrulewidth\fi%
\hrule\@height 1.0pt \@width\headwidth\vskip1pt	%上面线为1pt粗  
\hrule\@height 0.5pt\@width\headwidth  			%下面0.5pt粗            
\vskip-2\headrulewidth\vskip-1pt}      			%两条线的距离1pt        
 \vspace{6mm}}     								%双线与下面正文之间的垂直间距              
\makeatother  

%%%%%%%%%%%%%%%%%%%%%%%%%%%%%%%%%%%%%%%%%%%%%%
\numberwithin{equation}{section}
%\usepackage[thmmarks, amsmath, thref]{ntheorem}
\newtheorem{myThm}{myThm}
\newtheorem*{myDef}{Definition}
\newtheorem*{mySol}{Solution}
\newtheorem*{myProof}{Proof}
\newtheorem*{myRemark}{备注}
\renewcommand{\tilde}{\widetilde}
\renewcommand{\hat}{\widehat}
\newcommand{\indep}{\rotatebox[origin=c]{90}{$\models$}}
\newcommand*\diff{\mathop{}\!\mathrm{d}}

\usepackage{multirow}

%--

%--
\begin{document}

\title{数字信号处理\\
    作业四}
\author{你的名字\, 你的学号}
\maketitle
%%%%%%%% 注意: 使用XeLatex 编译可能会报错,请使用 pdfLaTex 编译 %%%%%%%

\section*{作业提交注意事项}
\begin{tcolorbox}
    \begin{enumerate}
        \item[(1)] 本次作业提交截止时间为~\textcolor{red}{\textbf{2023/12/18  23:59:59}},截止时间后不再接收作业,本次作业记零分        
        \item[(2)] 作业提交方式:使用此~\LaTeX~模板书写解答,只需提交编译生成的~pdf~文件,将~pdf~文件上传至\href{https://table.nju.edu.cn/dtable/forms/83126334-285c-4153-ba8a-cdc90d24358f/}{此~NJU Table};
        \item[(3)] pdf 文件命名方式:学号-姓名-作业号-v版本号, 例~ MG1900000-张三-1-v1;如果需要更改已提交的解答,请在截止时间之前提交新版本的解答,并将版本号加一;
        \item[(4)] 未按照要求提交作业,或~pdf~命名方式不正确,将会被扣除部分作业分数。

    \end{enumerate}
\end{tcolorbox}


\newpage
\section{[60pts] 离散信号的傅里叶变换}
\begin{enumerate}
	\item 已知 $x[n]$ 经离散时间傅里叶变换后为 $X(e^{j\omega}) = \frac{\pi}{4} \delta(\omega - \frac{\pi}{4}) + \frac{\pi}{4} \delta(\omega + \frac{\pi}{4})$,求 $x[n]$;
	\item 求周期为 4 的序列$ x[n]=\{\cdots, 14, 12, 6, 10, \cdots\} $的傅里叶级数的系数;
	\item 计算下列有限长度序列的离散傅里叶变换(假设长度为 $N$):
	\begin{enumerate}
		\item[(1)] $x[n]=\delta[n-n_0],\ 0<n_0<N$;
		\item[(2)] $x[n]=a^n,\ 0\le n\le N-1$.
	\end{enumerate}
\end{enumerate}

\begin{framed}
    \begin{spacing}{1.5}
        \begin{itemize}
            \item 你的答案。
        \end{itemize}
    \end{spacing}
\end{framed}


\newpage
\section{[40pts] DTFT 和 DFS 的比较}
\noindent 已知 $x[n]=\{4,2,1,2,3\}$
\begin{itemize}
	\item[(1)]计算 $X({e}^{{j}\omega})=\operatorname{DTFT}(x[n])$ 及 $X[k]=\operatorname{DFT}(x[n])$.
	\item[(2)]将 $x[n]$ 的尾部补零, 得到 $x_0[n]=\{4,2,1,2,3,0,0,0\}$. 计算 $X_0({e}^{j \omega})=\operatorname{DTFT}(x_0[n])$ 及 $X_0[k]=\operatorname{DFT}(x_0[n])$.
	\item[(3)]将 (1), (2) 的结果加以比较, 得出相应的结论.
\end{itemize}





\begin{framed}
	\begin{spacing}{1.5}
		\begin{itemize}
			\item 你的答案。
		\end{itemize}
	\end{spacing}
\end{framed}






\end{document}