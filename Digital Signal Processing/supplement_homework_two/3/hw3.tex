\documentclass[tikz,a4paper,UTF8]{article}
\usepackage{ctex}
\usepackage[margin=1.25in]{geometry}
\usepackage{color}
\usepackage{graphicx}
\usepackage{amssymb}
\usepackage{amsmath}
\usepackage{amsthm}
\usepackage{enumerate}
\usepackage{bm}
\usepackage{hyperref}
\usepackage{pgfplots}
\usepackage{epsfig}
\usepackage{color}
\usepackage{tcolorbox}
\usepackage{mdframed}
\usepackage{lipsum}
\usepackage{framed}
\usepackage{setspace}
\usepackage{hyperref}
\usepackage{pgfplots}
\pgfplotsset{compat=newest}

\newmdtheoremenv{thm-box}{myThm}
\newmdtheoremenv{prop-box}{Proposition}
\newmdtheoremenv{def-box}{定义}

\setlength{\evensidemargin}{.25in}
\setlength{\textwidth}{6in}
\setlength{\topmargin}{-0.5in}
\setlength{\topmargin}{-0.5in}
% \setlength{\textheight}{9.5in}
%%%%%%%%%%%%%%%%%%此处用于设置页眉页脚%%%%%%%%%%%%%%%%%%
\usepackage{fancyhdr}                                
\usepackage{lastpage}                                           
\usepackage{layout}                                             
\footskip = 10pt 
\pagestyle{fancy}                    % 设置页眉                 
\lhead{2023年秋季}                    
\chead{数字信号处理}                                                
% \rhead{第\thepage/\pageref{LastPage}页} 
\rhead{作业三}                                                                                               
\cfoot{\thepage}                                                
\renewcommand{\headrulewidth}{1pt}  			%页眉线宽,设为0可以去页眉线
\setlength{\skip\footins}{0.5cm}    			%脚注与正文的距离           
\renewcommand{\footrulewidth}{0pt}  			%页脚线宽,设为0可以去页脚线

\makeatletter 									%设置双线页眉                                        
\def\headrule{{\if@fancyplain\let\headrulewidth\plainheadrulewidth\fi%
\hrule\@height 1.0pt \@width\headwidth\vskip1pt	%上面线为1pt粗  
\hrule\@height 0.5pt\@width\headwidth  			%下面0.5pt粗            
\vskip-2\headrulewidth\vskip-1pt}      			%两条线的距离1pt        
 \vspace{6mm}}     								%双线与下面正文之间的垂直间距              
\makeatother  

%%%%%%%%%%%%%%%%%%%%%%%%%%%%%%%%%%%%%%%%%%%%%%
\numberwithin{equation}{section}
%\usepackage[thmmarks, amsmath, thref]{ntheorem}
\newtheorem{myThm}{myThm}
\newtheorem*{myDef}{Definition}
\newtheorem*{mySol}{Solution}
\newtheorem*{myProof}{Proof}
\newtheorem*{myRemark}{备注}
\renewcommand{\tilde}{\widetilde}
\renewcommand{\hat}{\widehat}
\newcommand{\indep}{\rotatebox[origin=c]{90}{$\models$}}
\newcommand*\diff{\mathop{}\!\mathrm{d}}

\usepackage{multirow}

%--

%--
\begin{document}

\title{数字信号处理\\
    作业三}
\author{你的名字\, 你的学号}
\maketitle
%%%%%%%% 注意: 使用XeLatex 编译可能会报错,请使用 pdfLaTex 编译 %%%%%%%

\section*{作业提交注意事项}
\begin{tcolorbox}
    \begin{enumerate}
        \item[(1)] 本次作业提交截止时间为~\textcolor{red}{\textbf{2023/11/27  23:59:59}},截止时间后不再接收作业,本次作业记零分        
        \item[(2)] 作业提交方式:使用此~\LaTeX~模板书写解答,只需提交编译生成的~pdf~文件,将~pdf~文件上传至\href{https://table.nju.edu.cn/dtable/forms/b6525acd-3a73-44e0-bee4-ab17bf98f172/}{此~NJU Table};
        \item[(3)] pdf 文件命名方式:学号-姓名-作业号-v版本号, 例~ MG1900000-张三-1-v1;如果需要更改已提交的解答,请在截止时间之前提交新版本的解答,并将版本号加一;
        \item[(4)] 未按照要求提交作业,或~pdf~命名方式不正确,将会被扣除部分作业分数。

    \end{enumerate}
\end{tcolorbox}


\newpage
\section{[28pts] 连续信号的傅里叶变换}
\begin{enumerate}[(1)]
    \item 求 $x(t)=\left\{\begin{array}{rl}2-t, & 0 \leq t \leq 2 \\ 0, & t \text { 为其它值 }\end{array}\right.$ 的傅里叶变换 $X(j \Omega)$;
    \item 求 $y(t)=\left\{\begin{array}{rl}-t, & -2 \leq t \leq 0 \\ 0, & t \text { 为其它值 }\end{array}\right.$ 的傅里叶变换 $Y(j \Omega)$;
    \item $y(t)$ 可以看成由 $x(t)$ 在时域上平移所得, 请据此由 (1) (2) 两问的运算结果直接写出傅里叶变换的时移特性。
\end{enumerate}

\begin{framed}
    \begin{spacing}{1.5}
        \begin{itemize}
            \item 你的答案。
        \end{itemize}
    \end{spacing}
\end{framed}


\newpage
\section{[40pts] 系统的性质 }
设 $X(j\Omega)$ 是下图所示信号 $x(t)$ 的频谱,试在不计算 $X(j\Omega)$ 具体表达式的情况下完成以下计算:

\begin{figure}[h]
	\centering
	\begin{tikzpicture}[scale=0.8]
		\begin{axis}[
			axis lines=middle,
			axis equal,
			grid=minor,
			xmin=-2, xmax=4,
			ymin=0, ymax=3,
			xlabel=$t$, ylabel=$x(t)$,
			xtick={-1, 0, 1, 2, 3},
			ytick={1, 2},
			yticklabels={1, 2},
			xticklabels={-1, 0, 1, 2, 3},
			xlabel style={at={(axis description cs:1,0.18)},anchor=north},
			ylabel style={at={(axis description cs:0.33,0.95)},anchor=east},
			]
			
			\addplot+[black, no marks, domain=-1:0, samples=100]{2+2*x};
			\addplot+[black, no marks, domain=0:2, samples=100]{2};
			\addplot+[black, no marks, domain=2:3, samples=100]{6-2*x};
		\end{axis}
	\end{tikzpicture}
	\label{fig1}
\end{figure}

\begin{enumerate}[(1)]
	\item $X(0)$
	\item $\displaystyle\int^{\infty}_{-\infty}X(j\Omega)d\Omega$
	\item $\displaystyle\int^{\infty}_{-\infty}X(j\Omega)\frac{2\sin\Omega}{\Omega}e^{j3\Omega}d\Omega$
	\item $\displaystyle\int^{\infty}_{-\infty}\left|X(j\Omega)\right|^2d\Omega$
\end{enumerate}





\begin{framed}
	\begin{spacing}{1.5}
		\begin{itemize}
			\item 你的答案。
		\end{itemize}
	\end{spacing}
\end{framed}




\newpage
\section{[32pts] 信号的采样}
\begin{enumerate}
	\item 对于实数信号~$x(t)$,已知用频率~$\omega_s=1500 \pi \text{ rad/s}$~采样时,信号可由其样本点唯一确定,请说明当~$\omega$~满足什么条件时能够使~$X(j\omega)=0$;
	\item 计算下列信号的奈奎斯特频率
	\begin{enumerate}
		\item $x(t)=\frac{\sin(2000\pi t)}{\pi t}$;
		\item $x(t)=x_1 \bigl(\frac{t}{2}\bigr) x_2 (2t)$,其中,$x_1 (t)$ 的最高频率为 150Hz,$x_2 (t)$ 的最高频率为 100Hz.
	\end{enumerate}
\end{enumerate}

\begin{framed}
    \begin{spacing}{1.5}
        \begin{itemize}
            \item 你的答案。
        \end{itemize}
    \end{spacing}
\end{framed}




\end{document}