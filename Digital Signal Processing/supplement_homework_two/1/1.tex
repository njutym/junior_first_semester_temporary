\documentclass[a4paper,UTF8]{article}
\usepackage{ctex}
\usepackage[margin=1.25in]{geometry}
\usepackage{color}
\usepackage{graphicx}
\usepackage{amssymb}
\usepackage{amsmath}
\usepackage{amsthm}
\usepackage{enumerate}
\usepackage{bm}
\usepackage{hyperref}
\usepackage{pgfplots}
\usepackage{epsfig}
\usepackage{color}
\usepackage{tcolorbox}
\usepackage{mdframed}
\usepackage{lipsum}
\usepackage{framed}
\usepackage{setspace}

\newmdtheoremenv{thm-box}{myThm}
\newmdtheoremenv{prop-box}{Proposition}
\newmdtheoremenv{def-box}{定义}

\setlength{\evensidemargin}{.25in}
\setlength{\textwidth}{6in}
\setlength{\topmargin}{-0.5in}
\setlength{\topmargin}{-0.5in}
% \setlength{\textheight}{9.5in}
%%%%%%%%%%%%%%%%%%此处用于设置页眉页脚%%%%%%%%%%%%%%%%%%
\usepackage{fancyhdr}                                
\usepackage{lastpage}                                           
\usepackage{layout}                                             
\footskip = 10pt 
\pagestyle{fancy}                    % 设置页眉                 
\lhead{2023年秋季}                    
\chead{数字信号处理}                                                
% \rhead{第\thepage/\pageref{LastPage}页} 
\rhead{作业一}                                                                                               
\cfoot{\thepage}                                                
\renewcommand{\headrulewidth}{1pt}  			%页眉线宽,设为0可以去页眉线
\setlength{\skip\footins}{0.5cm}    			%脚注与正文的距离           
\renewcommand{\footrulewidth}{0pt}  			%页脚线宽,设为0可以去页脚线

\makeatletter 									%设置双线页眉                                        
\def\headrule{{\if@fancyplain\let\headrulewidth\plainheadrulewidth\fi%
\hrule\@height 1.0pt \@width\headwidth\vskip1pt	%上面线为1pt粗  
\hrule\@height 0.5pt\@width\headwidth  			%下面0.5pt粗            
\vskip-2\headrulewidth\vskip-1pt}      			%两条线的距离1pt        
 \vspace{6mm}}     								%双线与下面正文之间的垂直间距              
\makeatother  

%%%%%%%%%%%%%%%%%%%%%%%%%%%%%%%%%%%%%%%%%%%%%%
\numberwithin{equation}{section}
%\usepackage[thmmarks, amsmath, thref]{ntheorem}
\newtheorem{myThm}{myThm}
\newtheorem*{myDef}{Definition}
\newtheorem*{mySol}{Solution}
\newtheorem*{myProof}{Proof}
\newtheorem*{myRemark}{备注}
\renewcommand{\tilde}{\widetilde}
\renewcommand{\hat}{\widehat}
\newcommand{\indep}{\rotatebox[origin=c]{90}{$\models$}}
\newcommand*\diff{\mathop{}\!\mathrm{d}}

\usepackage{multirow}

%--

%--
\begin{document}

\title{数字信号处理\\
    作业一}
\author{你的名字\, 你的学号}
\maketitle
%%%%%%%% 注意: 使用XeLatex 编译可能会报错,请使用 pdfLaTex 编译 %%%%%%%

\section*{作业提交注意事项}
\begin{tcolorbox}
    \begin{enumerate}
        \item[(1)] 本次作业提交截止时间为~\textcolor{red}{\textbf{2023/10/15  23:59:59}},截止时间后不再接收作业,本次作业记零分;
        \item[(2)] 作业提交方式:使用此~LaTex~模板书写解答,只需提交编译生成的~pdf~文件,将~pdf~文件上传至https://box.nju.edu.cn/u/d/dccd9cb71a344d52a9f1/;
        \item[(3)] pdf 文件命名方式:学号-姓名-作业号-v版本号, 例~ MG1900000-张三-1-v1;如果需要更改已提交的解答,请在截止时间之前提交新版本的解答,并将版本号加一;
        \item[(4)] 未按照要求提交作业,或~pdf~命名方式不正确,将会被扣除部分作业分数。

    \end{enumerate}
\end{tcolorbox}


\newpage
\section{[15pts] 信号的周期性}
判断下列信号的周期性,并回答\textbf{是}、\textbf{否},结果可能需要分类讨论。如果是周期信号,请给出其最小正周期。
\begin{enumerate}[(1)]
    \item $x(t)=\displaystyle\sin^22t+\cos\frac{\pi}{2} t$
    \item $x(t)=\displaystyle\frac{\sin t + 2\sin2t + \sin3t + \sin 4t}{\cos t}$
    \item $x(n)=\sin 3kn+\cos 4kn$, $k$为某一正实数。
\end{enumerate}

\begin{framed}
    \begin{spacing}{1.5}
        \begin{itemize}
            \item 你的答案。
        \end{itemize}
    \end{spacing}
\end{framed}


\newpage
\section{[10pts] 信号的模长}
求以下信号的模长并画出示意图:
\begin{equation*}
    \begin{aligned}
        x(t)=e^{j4t}+e^{j7t}
    \end{aligned}
\end{equation*}

\begin{framed}
    \begin{spacing}{1.5}
        \begin{itemize}
            \item 你的答案。
        \end{itemize}
    \end{spacing}
\end{framed}


\newpage
\section{[15pts] 系统的性质 }
判断下列系统的线性/非线性、时变性/时不变性、可逆性/不可逆性和因果性/非因果性:
\begin{enumerate}[(1)]
    \item $y(t)=\displaystyle\frac{(x(t)+x^{\prime}(t))^2}{x(t)}$
    \item $y(t) +y^{\prime}(t)=x(t)+x^{\prime}(t+1)-x^{\prime}(t-1)$
    \item $y(n)=\displaystyle\sum^{\infty}_{k=1}\frac{(\ln x(n))^k}{k!}$
\end{enumerate}

\begin{framed}
    \begin{spacing}{1.5}
        \begin{itemize}
            \item 你的答案。
        \end{itemize}
    \end{spacing}
\end{framed}


\newpage
\section{[20pts] 连续信号的性质与变换}
已知信号
\begin{equation*}
    \begin{aligned}
        x(t)=\left\{
        \begin{aligned}
             & t+2,  &  & t\in[-2,-1] \\
             & 1,    &  & t\in[-1,2]  \\
             & -t+2, &  & t\in[1,\frac{3}{2}]   \\
             & 0   , &  & \text{other}
        \end{aligned}
        \right.
    \end{aligned}
\end{equation*}
\begin{enumerate}[(1)]
    \item 求$x^{\prime}(t)-x^{\prime\prime}(t)$的表达式,并给出其大致图像。(冲激偶函数用$\delta^{\prime}(t)$表示,其图像为原点向y轴正负半轴分别延伸的箭头)
\end{enumerate}

\begin{framed}
    \begin{spacing}{1.5}
        \begin{itemize}
            \item 你的答案。
        \end{itemize}
    \end{spacing}
\end{framed}


\newpage
\section{[40pts] 卷积的计算 }
计算下列各小题的结果:
\begin{enumerate}[(1)]
    \item 设
          \begin{equation*}
              \begin{aligned}
                  x(t)=\left\{
                  \begin{aligned}
                       & 2t+1, &  & t\in[-1,1] \\
                       & 0,    &  & other
                  \end{aligned}
                  \right.
              \end{aligned}
          \end{equation*}
          试求$x(t)*x(t)$的结果。
    \item 求$y(t)=[2e^{-2(t+1)}u(t-1)]*[3e^{-3t}u(t)]$的表达式。
\end{enumerate}

\begin{framed}
    \begin{spacing}{1.5}
        \begin{itemize}
            \item 你的答案。
        \end{itemize}
    \end{spacing}
\end{framed}
\newpage
\end{document}